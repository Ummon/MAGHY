%% Réalisé à partir du template IEEE :
%% http://www.michaelshell.org/tex/ieeetran/
%% http://www.ctan.org/tex-archive/macros/latex/contrib/IEEEtran/
%% and
%% http://www.ieee.org/

\documentclass[journal, a4paper, frenchb]{IEEEtran}

\usepackage[francais]{babel}
\usepackage[utf8]{inputenc}
\usepackage[T1]{fontenc}
\usepackage{lmodern}
\usepackage{multicol}
\usepackage{color}
\usepackage{listings}
\usepackage{cite}

\usepackage[pdftex]{graphicx}
\graphicspath{./img/}
\DeclareGraphicsExtensions{.pdf,.jpeg,.png}

\usepackage{array}
\usepackage{mdwmath}
\usepackage{mdwtab}
\usepackage{eqparbox}
\usepackage{fixltx2e}
\usepackage{url}
% \usepackage{stfloats}

\definecolor{bluekeywords}{rgb}{0.13,0.13,1}
\definecolor{greencomments}{rgb}{0,0.5,0}
\definecolor{turqusnumbers}{rgb}{0.17,0.57,0.69}
\definecolor{redstrings}{rgb}{0.5,0,0}

\lstdefinelanguage{FSharp}
                {morekeywords={let, new, match, with, rec, open, module, namespace, type, of, member, and, for, in, do, begin, end, fun, function, try, mutable, if, then, else},
    keywordstyle=\color{bluekeywords},
    sensitive=false,
    morecomment=[l][\color{greencomments}]{///},
    morecomment=[l][\color{greencomments}]{//},
    morecomment=[s][\color{greencomments}]{{(*}{*)}},
    morestring=[b]",
    stringstyle=\color{redstrings}
    }
    
\def\IEEEkeywordsname{Mots-clefs}

\begin{document}

\title{Test-driven developement et pair programming}

\author{Étienne~Frank, Grégory Burri}

\markboth{Test-driven development et pair programming}
{Shell \MakeLowercase{\textit{et al.}}: Bare Demo of IEEEtran.cls for Journals}

\maketitle

\begin{abstract}
Cet article à pour but de présenter les approches \emph{TDD} (\emph{test-driven development}) et \emph{pair-programming} dans le domaine du développement logiciel, de mettre en avant leurs atouts ainsi que de les inscrire dans des méthodologies de développement connues.
\end{abstract}

\begin{IEEEkeywords}
TDD, test-driven development, pair-programming.
\end{IEEEkeywords}

\IEEEpeerreviewmaketitle

\section{Introduction}

\IEEEPARstart{C}{eci} est un test.

\section{\emph{TDD} (\emph{test-driven development})}

\subsection{Motivations}

Lorsqu'un développeur se met au travail et choisi de réaliser une nouvelle fonctionnalité d'un logiciel il va, en premier lieu, réfléchir comment celle-ci va être implémentée et quels changements il va devoir réaliser dans la structure actuelle du programme. Le développeur va ensuite ajouter ou modifier des types et implémenter les fonctions nécessaires pour répondre au spécifications. Puis, finalement, il va faire en sorte que le tout compile et il va lancer l'application afin de tester que ce qu'il vient de réaliser fonctionne bien.

Les problèmes de cette façon de faire sont multiples. Tout d'abord il y a de forte chance que la personne se soit écartée des spécifications ou ait voulu trop en faire en créant plus de types que nécessaire ou trop de couches d'abstraction. De plus, les tests n'étant effectués qu'à la fin de l'implémentation, il y fort à parier que le résultat ne corresponde pas complètement aux spécifications et que des bugs subsistent. Si certains de ces derniers sont liés au design alors il est probable que des modifications assez importantes vont devoir être réalisées. Si les tests ne sont que manuels, il n'y aura rien qui facilite la validation du fonctionnement des fonctionnalités dans le futur.

Pour palier à ces problèmes le développeur peut adopter l'approche \emph{TDD} qui consiste à, dans un premier temps, écrire un test puis, ensuite, à écrire l'implémentation associée. Cela à pour objectif de décrire ce que l'on veut avant de réaliser l'implémentation y répondant exactement. De cette manière uniquement le code nécessaire à faire passer le test est écrit et pas plus.


\subsection{Fonctionnement}

Voici les trois étapes du processus.

\begin{enumerate}
   \item Écriture d'un test unitaire, le test doit échouer (être marqué rouge).
   \item Réalisation d'une implémentation minimale afin de faire passer le test au vert.
   \item \emph{Refactorer} l'implémentation tout en maintenant le test au vert.
\end{enumerate}

Lors de l'écriture du test il faut se focaliser sur l'interface et non anticiper l'implémentation. Il est également important, qu'à la première étape, le test ne passe pas afin d'être sûr que ce que l'on veut tester n'existe pas ou ne fonctionne pas encore.

Ce processus est itératif et va être réitéré pour chaque test jusqu'à aboutir à une implémentation complète. Il est possible que des tests écrits précédemment doivent être modifiés ou deviennent obsolètes au file de la réalisation des fonctionnalités, dans ces cas il ne faudra pas hésiter à les modifier ou à les supprimer.

Afin d'aider le développeur à structurer ce processus il est fortement conseillé de garder à jour une liste des éléments à tester (\emph{todo list}). Avant d'écrire le premier test, cette liste va être peupler avec tout ce que l'on pense pertinent à tester, puis va être mise à jour après chaque test.

Le code exécuté durant les tests ne doit réaliser aucune entrées-sorties comme, par exemple, accéder au système de fichier au réseau ou à une base de données. Cela a pour but d'obtenir des tests s'exécutant rapidement et ne dépendant pas de données externes. Afin de simuler une ressource externe il est nécessaire d'utiliser des objets \emph{mocks} qui vont posséder un comportement spécifique à ce que l'on souhaite tester et qui vont vérifier comment la ressource est accédée.




\subsection{Bénéfices}


* Aide à atteindre un meilleur design, plus modulaire, plus testable
* Documentation de manière systèmatique de l'utilisation de l'API
* Confiance (au moment d'écrire, par rapport au code déjà écrit)
* Oblige à avoir des tests pour toutes les fonctionnalités
* Meilleure qualité de manière générale
* Approche pas forcément toujours nécessaire, dépend de l'utilisation du code et de la qualité souhaitée

* Cohesion vs indpendant (testable)


\subsection{Correction de bugs}

La correction du bug peut aussi suivre une forme s'approchant du \emph{TDD}. Lorsqu'un bug doit être corrigé, l'on va d'abord ajouter un ou plusieurs tests qui vont échoués afin de mettre en évidence le bug. Le code incriminé va ensuite être corrigé afin que le test passe. Une phase de \emph{refactoring} peut

Cette approche est à la fois importante pour la documentation du bug corrigé mais également pour éviter toute régression ultérieure.


\subsection{Intégration avec une approche agile}

Les tests unitaires -> s'ajoute un processus d'intégration continue



\section{Pair programming}
\subsection{Concept}
Le pair programming consiste à écrire du code en étant deux personnes. Une ayant le rôle de
conducteur, celui qui écrit les lignes de code. Et l’autre ayant le rôle de navigateur, celui qui doit
diriger l’autre. Ces rôles sont échangés à un intervalle plus ou moins régulier. Il faut comprendre que
le pair programming s’adapte à ses utilisateurs. Donc il est rare de trouver des règles absolues sur le
déroulement.

Cependant la règle qui pourrait être qualifiée la plus importante est celle de l’échange entre les deux
personnes. Cet échange permet de synchroniser le groupe. Lorsque vous regardez quelqu’un coder
vous aurez tendance à lâcher le fil. Sauf si ce codeur vous indiques chaque chose qu’il va
entreprendre et vous demande de vous assurer qu’il va au bon endroit. Très vite une inertie se crée
et on aura même tendance à oublier de prendre une pause.

En étant toujours concentrer sur le code on devient vite fatigué. Alors il ne faut s’arrêter 2-4 minutes,
penser à autre chose, manger un fruit, ou faire un peu d’exercice. C’est très important et ça rejoint
l’aspect de ne pas faire d’heure supplémentaire dans le monde agile. Sinon on arrive trop vite au
burnout.

Il faut toujours que les deux personnes sachent où elles vont. Nous venons de le voir sur une courte
portée, mais aussi avec une portée plus grande. Lorsqu’on commence à travailler on définit une
tâche global qui est notre but (en général elle nous prendra 1h-2h). Le navigateur doit d’autant plus
bien garder en mémoire ce but. Si le codeur commence à écrire du code qui n’est pas en rapport
avec la tâche finale, alors c’est au navigateur de le lui l’indiquer. Une fois la tâche finale définie, il faut
définir une petite tâche qui doit prendre entre 5 à 10 minutes. Cela contribue à garder la
synchronisation entre les deux membres.

Si le navigateur remarque que le codeur écrire une fonction non optimale, il ne doit pas déranger
tout de suite le conducteur. Il faut laisser au conducteur le temps de finir sa fonction, puis ensuite on
lui indique qu’il y a une erreur ou que la fonction n’est pas complète. Cela a pour but de ne pas
arrêter en cours de créativité la personne qui code.

Le pair programming coute en heures 15\% de plus, et fait 15\% de bug en moins[p.86]. Cependant il
est fait plus rapidement et les bugs peuvent couter cher. Donc si on regarde non pas juste le cout de
développement, mais le cout du programme une fois livré, on remarque que le pair programming est
bénéfique. Voici un tableau résumant les couts :

\begin{table*}[!t]
\begin{center}
    \caption{An Example of a Table}
    \label{table_example}
    \begin{tabular}{| l | l | l|}
    \hline
    & \textbf{Individual} & \textbf{Collaborators}  \\ \hline
    Hours & 2’000 hours & 2’300 hours \\ \hline
    Developpement Time (T) & 2’000 hours (12 months) & 1’150 hours (7 months) \\ \hline
    Development Cost (I) & \$100’000 & \$115’000 \\ \hline
    Defect in Field (DF) & 293 & 249 \\
    &\begin{tabular}{| l | r |} \hline
    	\multicolumn{2}{|c|}{Discovery bugs by Year} \\ \hline
    	T + Year 1& 169 \\
    	T + Year 2& 81 \\
    	T + Year 3& 35 \\\hline
    \end{tabular}&
	\begin{tabular}{| l | r |} \hline
    	\multicolumn{2}{|c|}{Discovery bugs by Year} \\ \hline
    	T + Year 1& 143 \\
    	T + Year 2& 68 \\
    	T + Year 3& 30 \\\hline
    \end{tabular}\\ \hline
    Operation Cost(M) & (169*33*40)/1.10 + & (143*33*40)/1.10 + \\ 
    & (81*33*40)/1.102 + & (68*33*40)/1.102 +\\
    & (35*33*40)/1.103 = 325,874 & (30*33*40)/1.103 = 275,534\\ \hline
    Discount Rate (d) & 10\% (or 0.8\% monthly) & 10\% (or 0.8\% monthly) \\ \hline
    Present Value of Lifetime Costs & 325874/1.00812 + 100,000 & 275534/1.0087 + 115,000\\
    (PVC) & =\$396,158 & =\$375,586\\ \hline
    Difference & & \$20,572 \\ \hline
    \end{tabular}
\end{center}
\end{table*}

Donc dans ce cas on voit qu’il est bénéfique d’utiliser le pair programming. Il faut retenir que cet
exemple utilise toujours des moyennes prisent dans des statistiques américaines. Si par exemple
votre but n’est pas de maintenir les bugs une fois votre programme fini, le pair programming vous
couteras plus cher.

Le pair programming aide également pour l’échange du savoir. Lorsque qu’on regarde quelqu’un
coder on voit tout de suite ce qu’on ferrait différemment. Par exemple si votre collègue n’utilise pas
un raccourci clavier pratique ou alors qu’il fait des boucles for au lieu des foreach. Parfois ça peut
être un détail mais d’autre fois on remarque que l’autre personne a une grosse lacune et elle peut
tout de suite apprendre et progresser via vos remarques.

Faire des remarques est très délicat. Lorsqu’on fait du pair programming il faut savoir mettre son égo
de coté. Mais il faut aussi savoir faire des remarques constructives et non pas juste rabaisser votre
collègue. Certaines personnes se sont déjà fait virer pour un manque de tact.

Pour l’expérience des pairs, nous pouvons avoir 3 cas.

\begin{itemize}
  \item Junior - Junior
  \item Senior - Junior
  \item Senior - Senior
\end{itemize}

La pair Senior-Junior est celles où le transfert de savoir est la plus importante. S’il n’est pas possible
de faire ce type de pair, alors on peut faire les deux restantes. Mais il ne faut pas oublier qu’un junior
peut apprendre à un senior. Parfois certain senior rouillent.

\subsection{Variation des pairs}

% blog.pivotal.io/pivotal-labs/labs/pair-programming-matrix
Il est bien de varier les paires. Faire le maximum de combinaisons possibles avec chaque membre de
l’équipe. Cela optimise le transfert de savoir de toute l’équipe. Mais souvent avec le temps il y a des
affinités qui se créer et on va préférer un collègue plutôt qu’un autre. Une manière de résoudre ce
problème et d’avoir une matrice triangulaire. L’on peut y voir rapidement si les pairs sont
déséquilibré ou non. Un problème surgit quand l’équipe de développement est trop grande, cela
devient un peu trop complexe. Alors c’est un bon signe pour faire des équipes plus petites.

Pour ce qui est de la cadence de rotation, il existe plusieurs avis. Certains disent qu’il faut changer
tous les jours, d’autre chaque semaine et voir certain à chaque fin de projet. Il faut trouver la
cadence qui convient au mieux à celle de l’équipe.

\subsection{Review programming}

Le concept du review programming est de demander à une ou plusieurs personnes de relire votre
code et de donner des critiques constructives dessus. C’est une alternative si votre manager n’est pas
convaincu par le pair programming. Mais il se peut aussi que les reviewers ne soient pas très
impliqué et ne vous donne qu’un avis général. Le review programming a aussi une sorte de latence.
Lorsqu’on fait des critiques après que le code soit écrit, cela peut être moins bien perçu par le
codeur. Surtout quand c’est votre patron. Alors qu’en pair programming, on fait les remarques à
chaud, ce qui diminue l’impact de « tu n’as pas bien fait les choses ». Un des avantage de review,
c’est qu’on a pas besoin d’être dans la même pièce pour faire la review correctement.

Les deux méthodes ont un point faible, la motivation. Si un des développeurs n’est pas du tout
intéressé par le pair ou review programming, alors il est difficile d’en sortir quelque chose de bon.Même si l’initiateur tente de motiver son collègue. Ces techniques sont puissantes grâce à leur
aspect social, mais c’est aussi leur point faible.

\subsection{Remote programming}

Lorsqu’on travaille sur des projets avec des développeurs qui vivent dans d’autres pays, il est difficile
de se retrouver dans la même pièce pour programmer ensemble. La solution est de passer par
internet. Il faut un logiciel de voix sur IP(VoIP) et un logiciel de partage d’écran. Ces deux
technologies impliquent une bonne connexion internet pour ne pas avoir de latence lorsqu’on code
où qu’on regarde l’autre personne coder.

Voici une liste de logiciel VoIP (non exhaustive) :

\begin{multicols}{2}
\begin{itemize}
    \item Skype
    \item Hangouts (Google+)
    \item TeamSpeak
    \item Mumble
\end{itemize}
\end{multicols}

Pour ce qui est du partage d’écran nous avons trois choix possible : partager le bureau et son
contrôle, partager juste le bureau et partager juste l’IDE. Les solutions sont de la plus gourmande en
connexion à la moins gourmande.

\begin{table*}[!t]
\begin{center}
    \caption{An Example of a Table 2}
    \label{table_example_2}
    \begin{tabular}{| l | l | l|}
    \hline
    Partage controle & Partage bureau & Partage IDE  \\ \hline
    Screenhero* & Skype & Cloud9 \\
    TeamViewer & Hang outs & Tmux (vim, emacs) \\ 
    VNC &  & Floobits ( \\
     & &  \hspace{1mm}  sublimetext, IntelliJ)\\
     & & Motepair (Atome) \\
     & & Madeye \\
    \hline
    \end{tabular}
\end{center}
\end{table*}

\subsection{Problème des métriques}
% https://github.com/therubymug/hitch
% https://github.com/glg/git-pairing

Le pair programming pose des problèmes lorsqu’on a un environnement qui calcule des métriques.
Souvent ces derniers ne sont pas adaptés pour des pairs. Il existe néant moins des programmes
communautaires qui tentent de pallier ce problème. Par exemple git-pairing permet d’indiquer
qu’elles pairs étaient sur le commit. Cela fonctionne mais il y a peu d’utilisateur donc il faut être
attentif si jamais il y a des bugs.

\subsection{Pair programming dans l’éducation}

% http://www.realsearchgroup.org/pairlearning/index.php

Deux universités ont fait des statistiques sur leurs étudiants qui commencer dans leurs études. La
North Carolina State University (NCSU) et la University of California Santa Cruz (UCSC). L’étude
portait sur l’impact du pair programming dans les cours de programmation. Elle s’est déroulée sur
1200 élèves sur une période de 3 ans (bachelor).

Les étudiants ayant fait du pair programming avait de meilleures notes en travaux pratiques et écrits.
Il y a eu plus d’étudiants qui réussissaient leur examen de première avec une note de C ou meilleur
(équivalent au 5 suisse). Lorsqu’ils étaient en 2 ème année, les étudiants avaient tendance à mieux
maintenir voir à améliorer leurs moyennes du cours de programmation de l’année précédente. 77\%
des étudiants qui avaient fait du pair programming choisissaient en deuxième année la suite du cours(contre 62\%). Et avaient tendences à mieux réussir le cours de deuxième année que ceux qui avait
programmé seul. Et finalement plus d’étudiants choisissaient l’informatique comme matière
principale ( NCSU: 57~\% vs. 34~\% ; UCSC : 25~\% vs. 11~\% ).

Finalement le mythe que les professeurs peuvent avoir sur « seulement un seul étudiant apprend
lorsqu’il y a du pair programming » semble être faux. Les Etats-Unis manquent de programmeurs, et
donc ils ont fait une campagne pour promouvoir la programmation dans les écoles. Cette campagne
demandait à beaucoup de célébrités de promouvoir la programmation (Barack Obama, Mark
Zuckerberg, Shakira, ...). Et certains professeur trouvait plus ludique pour de jeune écolier
d’apprendre la programmation en pair plutôt que seul.

https://www.youtube.com/watch?v=vgkahOzFH2Q

\section{Cas pratique}

\subsection{But}

Afin d'évaluer ces deux approches de manière concrète nous les avons mis en œuvre à l'aide d'un petit cas pratique. Cet exemple ne se veut en aucun cas exhaustive et ne montre qu'une introduction échelle réduite.


\subsection{Outils utilisés}

\subsection{blah}


\begin{figure*}[!t]
\caption{Exemple de tests concernant le type \texttt{ISBN13}}
\label{fig_test_example_isbn13}
\begin{lstlisting}[language=FSharp, frame=single, basicstyle=\ttfamily\scriptsize]
[<TestFixture>]
type ``ISBN-13`` () =
    let validISBN = "9781784391232"
    let invalidISBN = "9781784391233"

    [<Test>]
    member this.``Creating an ISBN from a correct string should be OK`` () =
        ISBN13 validISBN |> string |> should equal validISBN

    [<Test>]
    member this.``Two ISBN objects with the same ISBN should be equal`` () = 
        ISBN13 validISBN |> should equal (ISBN13 validISBN)

    [<Test>]
    member this.``Trying to create an ISBN from an incorrect number should raise an exception`` () =
        (fun () -> ISBN13 invalidISBN |> ignore) |> should throw typeof<InvalidISBNString>

    [<Test>]
    member this.``Trying to create an ISBN from an empty string should raise an exception`` () =
        (fun () -> ISBN13 "" |> ignore) |> should throw typeof<InvalidISBNString>

    [<Test>]
    member this.``Trying to create an ISBN from a non-digits string should raise an exception`` () = 
        (fun () -> ISBN13 "abc;$" |> ignore) |> should throw typeof<InvalidISBNString>

    [<Test>]
    member this.``Trying to create an ISBN from a null string should raise an exception`` () =
        (fun () -> ISBN13 null |> ignore) |> should throw typeof<InvalidISBNString>
\end{lstlisting}
\end{figure*}




\section{Conclusion}

% Can use something like this to put references on a page
% by themselves when using endfloat and the captionsoff option.
\ifCLASSOPTIONcaptionsoff
  \newpage
\fi


\begin{thebibliography}{1}

\bibitem{IEEEhowto:kopka}
H.~Kopka and P.~W. Daly, \emph{A Guide to \LaTeX}, 3rd~ed.\hskip 1em plus
  0.5em minus 0.4em\relax Harlow, England: Addison-Wesley, 1999.

\end{thebibliography}


\begin{IEEEbiography}{Michael Shell}
Biography text here.
\end{IEEEbiography}

% if you will not have a photo at all:
\begin{IEEEbiographynophoto}{John Doe}
Biography text here.
\end{IEEEbiographynophoto}

% insert where needed to balance the two columns on the last page with
% biographies
%\newpage

\begin{IEEEbiographynophoto}{Jane Doe}
Biography text here.
\end{IEEEbiographynophoto}


% that's all folks
\end{document}
