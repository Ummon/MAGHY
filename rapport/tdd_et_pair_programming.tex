%% Réalisé à partir du template IEEE :
%% http://www.michaelshell.org/tex/ieeetran/
%% http://www.ctan.org/tex-archive/macros/latex/contrib/IEEEtran/
%% and
%% http://www.ieee.org/

\documentclass[journal, a4paper]{IEEEtran}

\usepackage[francais]{babel}
\usepackage[utf8]{inputenc}
\usepackage[T1]{fontenc}
\usepackage{lmodern}
\usepackage{multicol}

\usepackage{cite}

\usepackage[pdftex]{graphicx}
\graphicspath{./img/}
\DeclareGraphicsExtensions{.pdf,.jpeg,.png}


% *** ALIGNMENT PACKAGES ***
%
\usepackage{array}
% Frank Mittelbach's and David Carlisle's array.sty patches and improves
% the standard LaTeX2e array and tabular environments to provide better
% appearance and additional user controls. As the default LaTeX2e table
% generation code is lacking to the point of almost being broken with
% respect to the quality of the end results, all users are strongly
% advised to use an enhanced (at the very least that provided by array.sty)
% set of table tools. array.sty is already installed on most systems. The
% latest version and documentation can be obtained at:
% http://www.ctan.org/tex-archive/macros/latex/required/tools/


\usepackage{mdwmath}
\usepackage{mdwtab}
% Also highly recommended is Mark Wooding's extremely powerful MDW tools,
% especially mdwmath.sty and mdwtab.sty which are used to format equations
% and tables, respectively. The MDWtools set is already installed on most
% LaTeX systems. The lastest version and documentation is available at:
% http://www.ctan.org/tex-archive/macros/latex/contrib/mdwtools/


% IEEEtran contains the IEEEeqnarray family of commands that can be used to
% generate multiline equations as well as matrices, tables, etc., of high
% quality.


\usepackage{eqparbox}
% Also of notable interest is Scott Pakin's eqparbox package for creating
% (automatically sized) equal width boxes - aka "natural width parboxes".
% Available at:
% http://www.ctan.org/tex-archive/macros/latex/contrib/eqparbox/



% *** FLOAT PACKAGES ***
%
\usepackage{fixltx2e}
% fixltx2e, the successor to the earlier fix2col.sty, was written by
% Frank Mittelbach and David Carlisle. This package corrects a few problems
% in the LaTeX2e kernel, the most notable of which is that in current
% LaTeX2e releases, the ordering of single and double column floats is not
% guaranteed to be preserved. Thus, an unpatched LaTeX2e can allow a
% single column figure to be placed prior to an earlier double column
% figure. The latest version and documentation can be found at:
% http://www.ctan.org/tex-archive/macros/latex/base/



%\usepackage{stfloats}
% stfloats.sty was written by Sigitas Tolusis. This package gives LaTeX2e
% the ability to do double column floats at the bottom of the page as well
% as the top. (e.g., "\begin{figure*}[!b]" is not normally possible in
% LaTeX2e). It also provides a command:
%\fnbelowfloat
% to enable the placement of footnotes below bottom floats (the standard
% LaTeX2e kernel puts them above bottom floats). This is an invasive package
% which rewrites many portions of the LaTeX2e float routines. It may not work
% with other packages that modify the LaTeX2e float routines. The latest
% version and documentation can be obtained at:
% http://www.ctan.org/tex-archive/macros/latex/contrib/sttools/
% Documentation is contained in the stfloats.sty comments as well as in the
% presfull.pdf file. Do not use the stfloats baselinefloat ability as IEEE
% does not allow \baselineskip to stretch. Authors submitting work to the
% IEEE should note that IEEE rarely uses double column equations and
% that authors should try to avoid such use. Do not be tempted to use the
% cuted.sty or midfloat.sty packages (also by Sigitas Tolusis) as IEEE does
% not format its papers in such ways.


%\ifCLASSOPTIONcaptionsoff
%  \usepackage[nomarkers]{endfloat}
% \let\MYoriglatexcaption\caption
% \renewcommand{\caption}[2][\relax]{\MYoriglatexcaption[#2]{#2}}
%\fi
% endfloat.sty was written by James Darrell McCauley and Jeff Goldberg.
% This package may be useful when used in conjunction with IEEEtran.cls'
% captionsoff option. Some IEEE journals/societies require that submissions
% have lists of figures/tables at the end of the paper and that
% figures/tables without any captions are placed on a page by themselves at
% the end of the document. If needed, the draftcls IEEEtran class option or
% \CLASSINPUTbaselinestretch interface can be used to increase the line
% spacing as well. Be sure and use the nomarkers option of endfloat to
% prevent endfloat from "marking" where the figures would have been placed
% in the text. The two hack lines of code above are a slight modification of
% that suggested by in the endfloat docs (section 8.3.1) to ensure that
% the full captions always appear in the list of figures/tables - even if
% the user used the short optional argument of \caption[]{}.
% IEEE papers do not typically make use of \caption[]'s optional argument,
% so this should not be an issue. A similar trick can be used to disable
% captions of packages such as subfig.sty that lack options to turn off
% the subcaptions:
% For subfig.sty:
% \let\MYorigsubfloat\subfloat
% \renewcommand{\subfloat}[2][\relax]{\MYorigsubfloat[]{#2}}
% For subfigure.sty:
% \let\MYorigsubfigure\subfigure
% \renewcommand{\subfigure}[2][\relax]{\MYorigsubfigure[]{#2}}
% However, the above trick will not work if both optional arguments of
% the \subfloat/subfig command are used. Furthermore, there needs to be a
% description of each subfigure *somewhere* and endfloat does not add
% subfigure captions to its list of figures. Thus, the best approach is to
% avoid the use of subfigure captions (many IEEE journals avoid them anyway)
% and instead reference/explain all the subfigures within the main caption.
% The latest version of endfloat.sty and its documentation can obtained at:
% http://www.ctan.org/tex-archive/macros/latex/contrib/endfloat/
%
% The IEEEtran \ifCLASSOPTIONcaptionsoff conditional can also be used
% later in the document, say, to conditionally put the References on a 
% page by themselves.





% *** PDF, URL AND HYPERLINK PACKAGES ***
%
\usepackage{url}
% url.sty was written by Donald Arseneau. It provides better support for
% handling and breaking URLs. url.sty is already installed on most LaTeX
% systems. The latest version can be obtained at:
% http://www.ctan.org/tex-archive/macros/latex/contrib/misc/
% Read the url.sty source comments for usage information. Basically,
% \url{my_url_here}.





% *** Do not adjust lengths that control margins, column widths, etc. ***
% *** Do not use packages that alter fonts (such as pslatex).         ***
% There should be no need to do such things with IEEEtran.cls V1.6 and later.
% (Unless specifically asked to do so by the journal or conference you plan
% to submit to, of course. )


% correct bad hyphenation here
% \hyphenation{op-tical net-works semi-conduc-tor}


\begin{document}
%
% paper title
% can use linebreaks \\ within to get better formatting as desired
\title{Test-driven developement et pair programming}
%
%
% author names and IEEE memberships
% note positions of commas and nonbreaking spaces ( ~ ) LaTeX will not break
% a structure at a ~ so this keeps an author's name from being broken across
% two lines.
% use \thanks{} to gain access to the first footnote area
% a separate \thanks must be used for each paragraph as LaTeX2e's \thanks
% was not built to handle multiple paragraphs
%

\author{Étienne~Frank, Grégory Burri}

% note the % following the last \IEEEmembership and also \thanks - 
% these prevent an unwanted space from occurring between the last author name
% and the end of the author line. i.e., if you had this:
% 
% \author{....lastname \thanks{...} \thanks{...} }
%                     ^------------^------------^----Do not want these spaces!
%
% a space would be appended to the last name and could cause every name on that
% line to be shifted left slightly. This is one of those "LaTeX things". For
% instance, "\textbf{A} \textbf{B}" will typeset as "A B" not "AB". To get
% "AB" then you have to do: "\textbf{A}\textbf{B}"
% \thanks is no different in this regard, so shield the last } of each \thanks
% that ends a line with a % and do not let a space in before the next \thanks.
% Spaces after \IEEEmembership other than the last one are OK (and needed) as
% you are supposed to have spaces between the names. For what it is worth,
% this is a minor point as most people would not even notice if the said evil
% space somehow managed to creep in.



% The paper headers
\markboth{Test-driven development et pair programming}%
{Shell \MakeLowercase{\textit{et al.}}: Bare Demo of IEEEtran.cls for Journals}
% The only time the second header will appear is for the odd numbered pages
% after the title page when using the twoside option.
% 
% *** Note that you probably will NOT want to include the author's ***
% *** name in the headers of peer review papers.                   ***
% You can use \ifCLASSOPTIONpeerreview for conditional compilation here if
% you desire.




% If you want to put a publisher's ID mark on the page you can do it like
% this:
%\IEEEpubid{0000--0000/00\$00.00~\copyright~2007 IEEE}
% Remember, if you use this you must call \IEEEpubidadjcol in the second
% column for its text to clear the IEEEpubid mark.


% use for special paper notices
%\IEEEspecialpapernotice{(Invited Paper)}


% make the title area
\maketitle


\begin{abstract}
Cet article à pour but de présenter les approches \emph{TDD} (\emph{test-driven development}) et \emph{pair-programming} dans le domaine du développement logiciel, de mettre en avant leurs atouts ainsi que de les inscrire dans des méthodologies de développement connues.
\end{abstract}

\begin{IEEEkeywords}
TDD, test-driven development, pair-programming.
\end{IEEEkeywords}

\IEEEpeerreviewmaketitle

\section{Introduction}

\IEEEPARstart{C}{eci} est un test.

\section{\emph{TDD} (\emph{test-driven development})}

\subsection{Motivations}

Lorsqu'un développeur se met au travail et choisi de réaliser une nouvelle fonctionnalité d'un logiciel il va, en premier lieu, réfléchir comment celle-ci va être implémentée et quels changements il va devoir réaliser dans la structure actuelle du programme. Le développeur va ensuite ajouter ou modifier des types et implémenter les fonctions nécessaires pour répondre au spécifications. Puis, finalement, il va faire en sorte que le tout compile et il va lancer l'application afin de tester que ce qu'il vient de réaliser fonctionne bien.

Les problèmes de cette façon de faire sont multiples. Tout d'abord il y a de forte chance que la personne se soit écartée des spécifications ou ait voulu trop en faire en créant plus de types que nécessaire ou trop de couches d'abstraction. De plus, les tests n'étant effectués qu'à la fin de l'implémentation, il y fort à parier que le résultat ne corresponde pas complètement aux spécifications et que des bugs subsistent. Si certains de ces derniers sont liés au design alors il est probable que des modifications assez importantes vont devoir être réalisées. Si les tests ne sont que manuels, il n'y aura rien qui facilite la validation du fonctionnement des fonctionnalités dans le futur.

Pour palier à ces problèmes le développeur peut adopter l'approche \emph{TDD} qui consiste à, dans un premier temps, écrire un test puis, ensuite, à écrire l'implémentation associée. Cela à pour objectif de décrire ce que l'on veut avant de réaliser l'implémentation y répondant exactement. De cette manière l'on écrit uniquement le code nécessaire à faire passer le test et pas plus.


\subsection{Fonctionnement}





* Écrire un test -> il doit être rouge
* Écrire l'implémentation minimum -> le test passe au vert
* Refactorer l'implémentation 
* Au fure et à mesure que des tests vont être écrit d'autres vont apparaître dans l'esprit du développeur, 
* Processus organique


\subsection{Bénéfices}


* Documentation de l'utilisation de l'API
* Meilleur design
* Confiance (au moment d'écrire, par rapport au code déjà écrit)
* Oblige à avoir des tests pour toutes les fonctionnalités
* Meilleure qualité

* Approche pas forcément toujours nécessaire



\subsection{Correction de bugs}

La correction du bug peut aussi suivre une forme s'approchant du \emph{TDD}. Lorsqu'un bug doit être corrigé, l'on va d'abord ajouter un ou plusieurs tests qui vont échoués afin de mettre en évidence le bug. Le code incriminé va ensuite être corrigé afin que le test passe. Une phase de \emph{refactoring} peut

Cette approche est à la fois importante pour la documentation du bug corrigé mais également pour éviter toute régression ultérieure.


\subsection{Intégration avec une approche agile}

Les tests unitaires -> s'ajoute un processus d'intégration continue



\section{Pair programming}
\subsection{Concept}
Le pair programming consiste à écrire du code en étant deux personnes. Une ayant le rôle de
conducteur, celui qui écrit les lignes de code. Et l’autre ayant le rôle de navigateur, celui qui doit
diriger l’autre. Ces rôles sont échangés à un intervalle plus ou moins régulier. Il faut comprendre que
le pair programming s’adapte à ses utilisateurs. Donc il est rare de trouver des règles absolues sur le
déroulement.

Cependant la règle qui pourrait être qualifiée la plus importante est celle de l’échange entre les deux
personnes. Cet échange permet de synchroniser le groupe. Lorsque vous regardez quelqu’un coder
vous aurez tendance à lâcher le fil. Sauf si ce codeur vous indiques chaque chose qu’il va
entreprendre et vous demande de vous assurer qu’il va au bon endroit. Très vite une inertie se crée
et on aura même tendance à oublier de prendre une pause.

En étant toujours concentrer sur le code on devient vite fatigué. Alors il ne faut s’arrêter 2-4 minutes,
penser à autre chose, manger un fruit, ou faire un peu d’exercice. C’est très important et ça rejoint
l’aspect de ne pas faire d’heure supplémentaire dans le monde agile. Sinon on arrive trop vite au
burnout.

Il faut toujours que les deux personnes sachent où elles vont. Nous venons de le voir sur une courte
portée, mais aussi avec une portée plus grande. Lorsqu’on commence à travailler on définit une
tâche global qui est notre but (en général elle nous prendra 1h-2h). Le navigateur doit d’autant plus
bien garder en mémoire ce but. Si le codeur commence à écrire du code qui n’est pas en rapport
avec la tâche finale, alors c’est au navigateur de le lui l’indiquer. Une fois la tâche finale définie, il faut
définir une petite tâche qui doit prendre entre 5 à 10 minutes. Cela contribue à garder la
synchronisation entre les deux membres.

Si le navigateur remarque que le codeur écrire une fonction non optimale, il ne doit pas déranger
tout de suite le conducteur. Il faut laisser au conducteur le temps de finir sa fonction, puis ensuite on
lui indique qu’il y a une erreur ou que la fonction n’est pas complète. Cela a pour but de ne pas
arrêter en cours de créativité la personne qui code.

Le pair programming coute en heures 15\% de plus, et fait 15\% de bug en moins[p.86]. Cependant il
est fait plus rapidement et les bugs peuvent couter cher. Donc si on regarde non pas juste le cout de
développement, mais le cout du programme une fois livré, on remarque que le pair programming est
bénéfique. Voici un tableau résumant les couts :

\begin{center}
    \begin{tabular}{| l | l | l|}
    \hline
    & \textbf{Individual} & \textbf{Collaborators}  \\ \hline
    Hours & 2’000 hours & 2’300 hours \\ \hline
    Developpement Time (T) & 2’000 hours (12 months) & 1’150 hours (7 months) \\ \hline
    Development Cost (I) & \$100’000 & \$115’000 \\ \hline
    Defect in Field (DF) & 293 & 249 \\
    &\begin{tabular}{| l | r |} \hline
    	\multicolumn{2}{|c|}{Discovery bugs by Year} \\ \hline
    	T + Year 1& 169 \\
    	T + Year 2& 81 \\
    	T + Year 3& 35 \\\hline
    \end{tabular}&
	\begin{tabular}{| l | r |} \hline
    	\multicolumn{2}{|c|}{Discovery bugs by Year} \\ \hline
    	T + Year 1& 143 \\
    	T + Year 2& 68 \\
    	T + Year 3& 30 \\\hline
    \end{tabular}\\ \hline
    Operation Cost(M) & (169*33*40)/1.10 + & (143*33*40)/1.10 + \\ 
    & (81*33*40)/1.102 + & (68*33*40)/1.102 +\\
    & (35*33*40)/1.103 = 325,874 & (30*33*40)/1.103 = 275,534\\ \hline
    Discount Rate (d) & 10\% (or 0.8\% monthly) & 10\% (or 0.8\% monthly) \\ \hline
    Present Value of Lifetime Costs & 325874/1.00812 + 100,000 & 275534/1.0087 + 115,000\\
    (PVC) & =\$396,158 & =\$375,586\\ \hline
    Difference & & \$20,572 \\ \hline
    \end{tabular}
\end{center}

Donc dans ce cas on voit qu’il est bénéfique d’utiliser le pair programming. Il faut retenir que cet
exemple utilise toujours des moyennes prisent dans des statistiques américaines. Si par exemple
votre but n’est pas de maintenir les bugs une fois votre programme fini, le pair programming vous
couteras plus cher.

Le pair programming aide également pour l’échange du savoir. Lorsque qu’on regarde quelqu’un
coder on voit tout de suite ce qu’on ferrait différemment. Par exemple si votre collègue n’utilise pas
un raccourci clavier pratique ou alors qu’il fait des boucles for au lieu des foreach. Parfois ça peut
être un détail mais d’autre fois on remarque que l’autre personne a une grosse lacune et elle peut
tout de suite apprendre et progresser via vos remarques.

Faire des remarques est très délicat. Lorsqu’on fait du pair programming il faut savoir mettre son égo
de coté. Mais il faut aussi savoir faire des remarques constructives et non pas juste rabaisser votre
collègue. Certaines personnes se sont déjà fait virer pour un manque de tact.

Pour l’expérience des pairs, nous pouvons avoir 3 cas.

\begin{itemize}
  \item Junior - Junior
  \item Senior - Junior
  \item Senior - Senior
\end{itemize}

La pair Senior-Junior est celles où le transfert de savoir est la plus importante. S’il n’est pas possible
de faire ce type de pair, alors on peut faire les deux restantes. Mais il ne faut pas oublier qu’un junior
peut apprendre à un senior. Parfois certain senior rouillent.

\subsection{Varation des pairs}

blog.pivotal.io/pivotal-labs/labs/pair-programming-matrix
Il est bien de varier les paires. Faire le maximum de combinaisons possibles avec chaque membre de
l’équipe. Cela optimise le transfert de savoir de toute l’équipe. Mais souvent avec le temps il y a des
affinités qui se créer et on va préférer un collègue plutôt qu’un autre. Une manière de résoudre ce
problème et d’avoir une matrice triangulaire. L’on peut y voir rapidement si les pairs sont
déséquilibré ou non. Un problème surgit quand l’équipe de développement est trop grande, cela
devient un peu trop complexe. Alors c’est un bon signe pour faire des équipes plus petites.

Pour ce qui est de la cadence de rotation, il existe plusieurs avis. Certains disent qu’il faut changer
tous les jours, d’autre chaque semaine et voir certain à chaque fin de projet. Il faut trouver la
cadence qui convient au mieux à celle de l’équipe.

\subsection{Review programming}

Le concept du review programming est de demander à une ou plusieurs personnes de relire votre
code et de donner des critiques constructives dessus. C’est une alternative si votre manager n’est pas
convaincu par le pair programming. Mais il se peut aussi que les reviewers ne soient pas très
impliqué et ne vous donne qu’un avis général. Le review programming a aussi une sorte de latence.
Lorsqu’on fait des critiques après que le code soit écrit, cela peut être moins bien perçu par le
codeur. Surtout quand c’est votre patron. Alors qu’en pair programming, on fait les remarques à
chaud, ce qui diminue l’impact de « tu n’as pas bien fait les choses ». Un des avantage de review,
c’est qu’on a pas besoin d’être dans la même pièce pour faire la review correctement.

Les deux méthodes ont un point faible, la motivation. Si un des développeurs n’est pas du tout
intéressé par le pair ou review programming, alors il est difficile d’en sortir quelque chose de bon.Même si l’initiateur tente de motiver son collègue. Ces techniques sont puissantes grâce à leur
aspect social, mais c’est aussi leur point faible.

\subsection{Remote programming}

Lorsqu’on travaille sur des projets avec des développeurs qui vivent dans d’autres pays, il est difficile
de se retrouver dans la même pièce pour programmer ensemble. La solution est de passer par
internet. Il faut un logiciel de voix sur IP(VoIP) et un logiciel de partage d’écran. Ces deux
technologies impliquent une bonne connexion internet pour ne pas avoir de latence lorsqu’on code
où qu’on regarde l’autre personne coder.

Voici une liste de logiciel VoIP (non exhaustive) :

\begin{multicols}{2}
\begin{itemize}
    \item Skype
    \item Hangouts (Google+)
    \item TeamSpeak
    \item Mumble
\end{itemize}
\end{multicols}

Pour ce qui est du partage d’écran nous avons trois choix possible : partager le bureau et son
contrôle, partager juste le bureau et partager juste l’IDE. Les solutions sont de la plus gourmande en
connexion à la moins gourmande.

\begin{center}
    \begin{tabular}{| l | l | l|}
    \hline
    Partage controle & Partage bureau & Partage IDE  \\ \hline
    Screenhero* & Skype & Cloud9 \\
    TeamViewer & Hang outs & Tmux (vim, emacs) \\ 
    VNC &  & Floobits ( \\
     & &  \hspace{1mm}  sublimetext, IntelliJ)\\
     & & Motepair (Atome) \\
     & & Madeye \\
    \hline
    \end{tabular}
\end{center}

\subsection{Problème des métriques}
https://github.com/therubymug/hitch

https://github.com/glg/git-pairing

Le pair programming pose des problèmes lorsqu’on a un environnement qui calcule des métriques.
Souvent ces derniers ne sont pas adaptés pour des pairs. Il existe néant moins des programmes
communautaires qui tentent de pallier ce problème. Par exemple git-pairing permet d’indiquer
qu’elles pairs étaient sur le commit. Cela fonctionne mais il y a peu d’utilisateur donc il faut être
attentif si jamais il y a des bugs.

\subsection{Pair programming dans l’éducation}

http://www.realsearchgroup.org/pairlearning/index.php

Deux universités ont fait des statistiques sur leurs étudiants qui commencer dans leurs études. La
North Carolina State University (NCSU) et la University of California Santa Cruz (UCSC). L’étude
portait sur l’impact du pair programming dans les cours de programmation. Elle s’est déroulée sur
1200 élèves sur une période de 3 ans (bachelor).

Les étudiants ayant fait du pair programming avait de meilleures notes en travaux pratiques et écrits.
Il y a eu plus d’étudiants qui réussissaient leur examen de première avec une note de C ou meilleur
(équivalent au 5 suisse). Lorsqu’ils étaient en 2 ème année, les étudiants avaient tendance à mieux
maintenir voir à améliorer leurs moyennes du cours de programmation de l’année précédente. 77\%
des étudiants qui avaient fait du pair programming choisissaient en deuxième année la suite du cours(contre 62\%). Et avaient tendences à mieux réussir le cours de deuxième année que ceux qui avait
programmé seul. Et finalement plus d’étudiants choisissaient l’informatique comme matière
principale ( NCSU: 57\% vs. 34\% ; UCSC : 25\% vs. 11\% ).

Finalement le mythe que les professeurs peuvent avoir sur « seulement un seul étudiant apprend
lorsqu’il y a du pair programming » semble être faux. Les Etats-Unis manquent de programmeurs, et
donc ils ont fait une campagne pour promouvoir la programmation dans les écoles. Cette campagne
demandait à beaucoup de célébrités de promouvoir la programmation (Barack Obama, Mark
Zuckerberg, Shakira, ...). Et certains professeur trouvait plus ludique pour de jeune écolier
d’apprendre la programmation en pair plutôt que seul.

https://www.youtube.com/watch?v=vgkahOzFH2Q

\section{Cas pratique}

\subsection{But}

Afin d'évaluer ces deux approches de manière concrète nous les avons mis en œuvre à l'aide d'un petit cas pratique. Cet exemple ne se veut en aucun cas exhaustive et ne montre qu'une introduction échelle réduite.


\subsection{Outils utilisés}


% An example of a floating figure using the graphicx package.
% Note that \label must occur AFTER (or within) \caption.
% For figures, \caption should occur after the \includegraphics.
% Note that IEEEtran v1.7 and later has special internal code that
% is designed to preserve the operation of \label within \caption
% even when the captionsoff option is in effect. However, because
% of issues like this, it may be the safest practice to put all your
% \label just after \caption rather than within \caption{}.
%
% Reminder: the "draftcls" or "draftclsnofoot", not "draft", class
% option should be used if it is desired that the figures are to be
% displayed while in draft mode.
%
%\begin{figure}[!t]
%\centering
%\includegraphics[width=2.5in]{myfigure}
% where an .eps filename suffix will be assumed under latex, 
% and a .pdf suffix will be assumed for pdflatex; or what has been declared
% via \DeclareGraphicsExtensions.
%\caption{Simulation Results}
%\label{fig_sim}
%\end{figure}

% Note that IEEE typically puts floats only at the top, even when this
% results in a large percentage of a column being occupied by floats.


% An example of a double column floating figure using two subfigures.
% (The subfig.sty package must be loaded for this to work.)
% The subfigure \label commands are set within each subfloat command, the
% \label for the overall figure must come after \caption.
% \hfil must be used as a separator to get equal spacing.
% The subfigure.sty package works much the same way, except \subfigure is
% used instead of \subfloat.
%
%\begin{figure*}[!t]
%\centerline{\subfloat[Case I]\includegraphics[width=2.5in]{subfigcase1}%
%\label{fig_first_case}}
%\hfil
%\subfloat[Case II]{\includegraphics[width=2.5in]{subfigcase2}%
%\label{fig_second_case}}}
%\caption{Simulation results}
%\label{fig_sim}
%\end{figure*}
%
% Note that often IEEE papers with subfigures do not employ subfigure
% captions (using the optional argument to \subfloat), but instead will
% reference/describe all of them (a), (b), etc., within the main caption.


% An example of a floating table. Note that, for IEEE style tables, the 
% \caption command should come BEFORE the table. Table text will default to
% \footnotesize as IEEE normally uses this smaller font for tables.
% The \label must come after \caption as always.
%
%\begin{table}[!t]
%% increase table row spacing, adjust to taste
%\renewcommand{\arraystretch}{1.3}
% if using array.sty, it might be a good idea to tweak the value of
% \extrarowheight as needed to properly center the text within the cells
%\caption{An Example of a Table}
%\label{table_example}
%\centering
%% Some packages, such as MDW tools, offer better commands for making tables
%% than the plain LaTeX2e tabular which is used here.
%\begin{tabular}{|c||c|}
%\hline
%One & Two\\
%\hline
%Three & Four\\
%\hline
%\end{tabular}
%\end{table}


% Note that IEEE does not put floats in the very first column - or typically
% anywhere on the first page for that matter. Also, in-text middle ("here")
% positioning is not used. Most IEEE journals use top floats exclusively.
% Note that, LaTeX2e, unlike IEEE journals, places footnotes above bottom
% floats. This can be corrected via the \fnbelowfloat command of the
% stfloats package.



\section{Conclusion}








% if have a single appendix:
%\appendix[Proof of the Zonklar Equations]
% or
%\appendix  % for no appendix heading
% do not use \section anymore after \appendix, only \section*
% is possibly needed

% use appendices with more than one appendix
% then use \section to start each appendix
% you must declare a \section before using any
% \subsection or using \label (\appendices by itself
% starts a section numbered zero.)
%


\appendices
\section{Proof of the First Zonklar Equation}
Appendix one text goes here.

% you can choose not to have a title for an appendix
% if you want by leaving the argument blank
\section{}
Appendix two text goes here.


% use section* for acknowledgement
\section*{Acknowledgment}


The authors would like to thank...


% Can use something like this to put references on a page
% by themselves when using endfloat and the captionsoff option.
\ifCLASSOPTIONcaptionsoff
  \newpage
\fi



% trigger a \newpage just before the given reference
% number - used to balance the columns on the last page
% adjust value as needed - may need to be readjusted if
% the document is modified later
%\IEEEtriggeratref{8}
% The "triggered" command can be changed if desired:
%\IEEEtriggercmd{\enlargethispage{-5in}}

% references section

% can use a bibliography generated by BibTeX as a .bbl file
% BibTeX documentation can be easily obtained at:
% http://www.ctan.org/tex-archive/biblio/bibtex/contrib/doc/
% The IEEEtran BibTeX style support page is at:
% http://www.michaelshell.org/tex/ieeetran/bibtex/
%\bibliographystyle{IEEEtran}
% argument is your BibTeX string definitions and bibliography database(s)
%\bibliography{IEEEabrv,../bib/paper}
%
% <OR> manually copy in the resultant .bbl file
% set second argument of \begin to the number of references
% (used to reserve space for the reference number labels box)
\begin{thebibliography}{1}

\bibitem{IEEEhowto:kopka}
H.~Kopka and P.~W. Daly, \emph{A Guide to \LaTeX}, 3rd~ed.\hskip 1em plus
  0.5em minus 0.4em\relax Harlow, England: Addison-Wesley, 1999.

\end{thebibliography}

% biography section
% 
% If you have an EPS/PDF photo (graphicx package needed) extra braces are
% needed around the contents of the optional argument to biography to prevent
% the LaTeX parser from getting confused when it sees the complicated
% \includegraphics command within an optional argument. (You could create
% your own custom macro containing the \includegraphics command to make things
% simpler here.)
%\begin{biography}[{\includegraphics[width=1in,height=1.25in,clip,keepaspectratio]{mshell}}]{Michael Shell}
% or if you just want to reserve a space for a photo:

\begin{IEEEbiography}{Michael Shell}
Biography text here.
\end{IEEEbiography}

% if you will not have a photo at all:
\begin{IEEEbiographynophoto}{John Doe}
Biography text here.
\end{IEEEbiographynophoto}

% insert where needed to balance the two columns on the last page with
% biographies
%\newpage

\begin{IEEEbiographynophoto}{Jane Doe}
Biography text here.
\end{IEEEbiographynophoto}

% You can push biographies down or up by placing
% a \vfill before or after them. The appropriate
% use of \vfill depends on what kind of text is
% on the last page and whether or not the columns
% are being equalized.

%\vfill

% Can be used to pull up biographies so that the bottom of the last one
% is flush with the other column.
%\enlargethispage{-5in}



% that's all folks
\end{document}
